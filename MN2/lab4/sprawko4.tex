\documentclass[]{article}
\usepackage{polski}
\usepackage[utf8]{inputenc}
\usepackage{mathtools}
\usepackage{pgfplots}
\usepackage{listings}
\usepackage{enumitem}
\usepackage{mathtools}
\usepackage{amsmath}
\usepackage{amssymb}
\usepackage{pgf,tikz}
\usepackage{mathrsfs}
\usetikzlibrary{arrows}
%opening
\let\normalint\int % PS
\def\int{\displaystyle\normalint} %PS
\title{\textbf{ Metody Numeryczne 2\\Laboratorium 4}\\
Aproksymacja średniokwadratowa ciągła w przestrzeni $L_p^2(-1,1)$ dla $p(x)=\frac{1}{\sqrt{(1-x^2)}}$ w bazie wielomianów Czebyszewa. }
\author{Szymon Adach}

\begin{document}

\maketitle


\section{Treść zadania}
 \textbf{Zadanie 3:} Aproksymacja średniokwadratowa ciągła w przestrzeni $L_p^2(-1,1)$ dla $p(x)=\frac{1}{\sqrt{(1-x^2)}}$ w bazie wielomianów Czebyszewa. Całkowanie n-punktową kwadraturą Gaussa-Czebyszewa. Tablicowanie funkcji, przybliżenia i błędu w m~punktach przedziału [-1, 1] oraz obliczanie błędu średniokwadratowego w tych punktach.
\section{Opis metody}
Metoda aproksymacji średniokwadratowej ciągłej dla funkcji $f(x)$polega na znalezieniu współczynników $\alpha_0...\alpha_n$ takich, że:\\ $f^*(x)=\alpha_0 \cdot P_0(x)+\alpha_1 \cdot P_1(x)+...+\alpha_n\cdot P_n(x)$, gdzie $P_i(x)$ to i-ty wielomian z bazy wielomianów Czebyszewa. Znalezienie współczynników $\alpha_i$ sprowadza się do rozwiązania układu równań liniowych $G\cdot\alpha=F$, gdzie i-ty wiersz wektora $F$ ma postać $\int_{-1}^{1}f(x)p(x)dx$. Z kolei macierz $G$ zdefiniowana jest następująco: 
\begin{center} $G = \frac{\pi}{2} \cdot
\begin{bmatrix}
2 &  &  & \\
& 1 &  &\\
& & 1 & \\
& & & \ddots \\
& & & & 1
\end{bmatrix}$ 
\end{center} 
\newpage
Całki w wektorze $F$ obliczane są za pomocą kwadaratury Gaussa-Czebyszewa:
\begin{center}
$\int_{-1}^{1}f(x)p(x)dx \approx \sum_{i=1}^{n}w_i\cdot f(x_i)$ \\
\end{center}
\begin{itemize}
\item $x_i$ - i-te miejsce zerowe wielomianu Czebyszewa określone wzorem:\\
$x_i=\cos(\frac{2i-1}{2n}\cdot \pi)$
\item$w_i$ - waga dana wzorem:
$w_i=\frac{\pi}{n}$\\
\end{itemize}
Wartości $f^*(x)=\alpha_0 \cdot P_0(x)+\alpha_1 \cdot P_1(x)+...+\alpha_n\cdot P_n(x)$ w punktach tablicowania obliczane są przy pomocy procedury \textbf{czebysz}, która korzysta ze wzoru:
\begin{center}
$P_n(x)=\frac{(x-\sqrt{x^2-1})^n+(x+\sqrt{x^2-1})^n}{2}$
\end{center}
Błąd średniokwadratowy obliczany jest ze wzoru:\\
\begin{center}
$e_sr=\sqrt{\frac{1}{n}\sum_{i=1}^{n}e_i^2}$ gdzie $e_i$ to błąd w i-tym punkcie tablicowania.
\end{center} 
\section{Działanie programu}

Program jest uruchamiany poleceniem:
\begin{center}
	\textbf{aproksymacja(func, m, n)}
\end{center} 
\begin{itemize}
	\item \textbf{func} - uchwyt do funkcji aproksymowanej
	\item \textbf{m} - liczba punktów do tablicowania
	\item \textbf{n} - liczba wielomianów Czebyszewa w bazie
\end{itemize}
Po wykonaniu obliczeń prezentowane są stablicowane wartości oraz wartość błędu średniokwadratowego w postaci:\\
$x$ \hspace{5mm} $f(x)$ \hspace{5mm} $f^*(x)$ \hspace{5mm} $err(x)$\\
Blad sredniokwadratowy: $e_sr$ \newpage
\section{Przykłady}
\begin{enumerate}
\item \textbf{Wywołanie:}
\verb|aproksymacja(@(x)x^6+5*x^5, 10, 6)|
\\\textbf{Wyjście:}\\
x\hspace{15mm}        f(x)\hspace{15mm}      f*(x)\hspace{13mm}     err(x)\\
-1.00000\hspace{5mm}  -4.00000\hspace{5mm}  -4.03125\hspace{5mm}   0.03125\\
-0.77778\hspace{5mm}  -1.20176\hspace{5mm}  -1.18324\hspace{5mm}   0.01852\\
-0.55556\hspace{5mm}  -0.23521\hspace{5mm}  -0.26408\hspace{5mm}   0.02887\\
-0.33333\hspace{5mm}  -0.01920\hspace{5mm}  -0.03331\hspace{5mm}   0.01410\\
-0.11111\hspace{5mm}  -0.00008\hspace{5mm}   0.02445\hspace{5mm}   0.02453\\
0.11111\hspace{5mm}   0.00009\hspace{6mm}   0.02462\hspace{7mm}   0.02453\\
0.33333\hspace{5mm}   0.02195\hspace{6mm}   0.00784\hspace{7mm}   0.01410\\
0.55556\hspace{5mm}   0.29401\hspace{6mm}   0.26514\hspace{7mm}   0.02887\\
0.77778\hspace{5mm}   1.64452\hspace{6mm}   1.66304\hspace{7mm}   0.01852\\
1.00000\hspace{5mm}   6.00000\hspace{6mm}   5.96875\hspace{7mm}   0.03125\\
Blad sredniokwadratowy: 0.024306

\item \textbf{Wywołanie:} \verb|aproksymacja(@(x)x^6+5*x^5, 10, 7)|
\\\textbf{Wyjście:}
    x\hspace{15mm}         f(x)\hspace{15mm}       f*(x)\hspace{15mm}      err(x)\\
    -1.00000\hspace{5mm}  -4.00000\hspace{5mm}  -4.00000\hspace{5mm}   0.00000\\
    -0.77778\hspace{5mm}  -1.20176\hspace{5mm}  -1.20176\hspace{5mm}   0.00000\\
    -0.55556\hspace{5mm}  -0.23521\hspace{5mm}  -0.23521\hspace{5mm}   0.00000\\
    -0.33333\hspace{5mm}  -0.01920 \hspace{3mm} -0.01920\hspace{6mm}   0.00000\\
    -0.11111\hspace{5mm}  -0.00008\hspace{5mm}  -0.00008\hspace{5mm}   0.00000\\
    0.11111 \hspace{5mm}  0.00009 \hspace{5mm}  0.00009 \hspace{5mm}  0.00000\\
    0.33333 \hspace{5mm}  0.02195\hspace{7mm}   0.02195\hspace{5mm}   0.00000\\
    0.55556 \hspace{5mm}  0.29401\hspace{7mm}   0.29401\hspace{5mm}   0.00000\\
    0.77778 \hspace{5mm}  1.64452 \hspace{5mm}  1.64452\hspace{6mm}   0.00000\\
    1.00000 \hspace{5mm}  6.00000 \hspace{5mm}  6.00000 \hspace{5mm}  0.00000\\
    Blad sredniokwadratowy: 0.000000
\item \textbf{Wywołanie:} \verb|aproksymacja(@cos, 10, 4)|
\\\textbf{Wyjście:}\\
x \hspace{15mm}        f(x)\hspace{15mm}       f*(x)\hspace{15mm}      err(x)\\
-1.00000 \hspace{5mm}   0.54030 \hspace{5mm}   0.53539 \hspace{5mm}   0.00491\\
-0.77778 \hspace{5mm}   0.71247 \hspace{5mm}   0.71697 \hspace{5mm}   0.00449\\
-0.55556 \hspace{5mm}   0.84961 \hspace{5mm}   0.85315 \hspace{5mm}   0.00354\\
-0.33333 \hspace{5mm}   0.94496 \hspace{5mm}   0.94394 \hspace{5mm}   0.00102\\
-0.11111 \hspace{5mm}   0.99383 \hspace{5mm}   0.98933 \hspace{5mm}   0.00450\\
0.11111  \hspace{6mm}  0.99383 \hspace{5mm}   0.98933 \hspace{5mm}   0.00450\\
0.33333  \hspace{6mm}  0.94496 \hspace{5mm}   0.94394 \hspace{5mm}   0.00102\\
0.55556  \hspace{6mm}  0.84961 \hspace{5mm}   0.85315 \hspace{5mm}   0.00354\\
0.77778  \hspace{6mm}  0.71247 \hspace{5mm}   0.71697 \hspace{5mm}   0.00449\\
1.00000  \hspace{6mm}  0.54030 \hspace{5mm}   0.53539 \hspace{5mm}   0.00491\\
Blad sredniokwadratowy: 0.003954
\end{enumerate}
\end{document}
